%% abtex2-modelo-trabalho-academico.tex, v-1.9.7 laurocesar
%% Copyright 2012-2018 by abnTeX2 group at http://www.abntex.net.br/ 
%%
%% This work may be distributed and/or modified under the
%% conditions of the LaTeX Project Public License, either version 1.3
%% of this license or (at your option) any later version.
%% The latest version of this license is in
%%   http://www.latex-project.org/lppl.txt
%% and version 1.3 or later is part of all distributions of LaTeX
%% version 2005/12/01 or later.
%%
%% This work has the LPPL maintenance status `maintained'.
%% 
%% The Current Maintainer of this work is the abnTeX2 team, led
%% by Lauro César Araujo. Further information are available on 
%% http://www.abntex.net.br/
%%
%% This work consists of the files abntex2-modelo-trabalho-academico.tex,
%% abntex2-modelo-include-comandos and abntex2-modelo-references.bib
%%

% ------------------------------------------------------------------------
% ------------------------------------------------------------------------
% abnTeX2: Modelo de Trabalho Academico (tese de doutorado, dissertacao de
% mestrado e trabalhos monograficos em geral) em conformidade com 
% ABNT NBR 14724:2011: Informacao e documentacao - Trabalhos academicos -
% Apresentacao
% ------------------------------------------------------------------------
% ------------------------------------------------------------------------

\documentclass[
	% -- opções da classe memoir --
	12pt,				% tamanho da fonte
	openright,			% capítulos começam em pág ímpar (insere página vazia caso preciso)
	oneside,			% para impressão em recto e verso. Oposto a oneside,twoside
	a4paper,			% tamanho do papel. 
	% -- opções da classe abntex2 --
	%chapter=TITLE,		% títulos de capítulos convertidos em letras maiúsculas
	%section=TITLE,		% títulos de seções convertidos em letras maiúsculas
	%subsection=TITLE,	% títulos de subseções convertidos em letras maiúsculas
	%subsubsection=TITLE,% títulos de subsubseções convertidos em letras maiúsculas
	% -- opções do pacote babel --
	english,			% idioma adicional para hifenização
	french,				% idioma adicional para hifenização
	spanish,			% idioma adicional para hifenização
	brazil				% o último idioma é o principal do documento
	]{abntex2}

% ---
% Pacotes básicos 
% ---
\usepackage{lmodern}			% Usa a fonte Latin Modern			
\usepackage[T1]{fontenc}		% Selecao de codigos de fonte.
\usepackage[utf8]{inputenc}		% Codificacao do documento (conversão automática dos acentos)
\usepackage{indentfirst}		% Indenta o primeiro parágrafo de cada seção.
\usepackage{color}				% Controle das cores
\usepackage{graphicx}			% Inclusão de gráficos
\usepackage{microtype} 			% para melhorias de justificação
% ---
\usepackage{titlesec}% just to generate text for the example
%
%Estilos dos capítulos
%
\titleformat{\chapter}[display]
  {\bfseries\Large}
  {\filright\MakeUppercase{\chaptertitlename} \Huge\thechapter}
  {1ex}
  {\titlerule\vspace{1ex}\filleft}
  [\vspace{1ex}\titlerule]
		
% ---
% estilos das partes
%\usepackage{psvectorian}

% Pacotes adicionais, usados apenas no âmbito do Modelo Canônico do abnteX2
% ---
\usepackage{lipsum}				% para geração de dummy text
% ---

% ---
% Pacotes de citações
% ---
\usepackage[brazilian,hyperpageref]{backref}	 % Paginas com as citações na bibl
\usepackage[alf]{abntex2cite}	% Citações padrão ABNT

% --- 
% CONFIGURAÇÕES DE PACOTES
% --- 

% ---
% Configurações do pacote backref
% Usado sem a opção hyperpageref de backref
\renewcommand{\backrefpagesname}{Citado na(s) página(s):~}
% Texto padrão antes do número das páginas
\renewcommand{\backref}{}
% Define os textos da citação
\renewcommand*{\backrefalt}[4]{
	\ifcase #1 %
		Nenhuma citação no texto.%
	\or
		Citado na página #2.%
	\else
		Citado #1 vezes nas páginas #2.%
	\fi}%
% ---

% ---
% Informações de dados para CAPA e FOLHA DE ROSTO
% ---
\titulo{Base de Conhecimento e Gestão do Conhecimento via Confluence}

% Nome Completo do academico 
\autor{Victor Hugo Vieira Cruz e Yago Fonseca Sarmento}
\local{Brasil}
\data{Novembro, 2025}
\instituicao{%
Instituto Federal de Educação, Ciência e Tecnologia de Goiás -- IFG
  \par}
 %Descomente a linha referebte ao seu tipo de curso
\tipotrabalho{Trabalho Acadêmico (Graduação)}
%\tipotrabalho{Monografia (Graduação)}
%\tipotrabalho{Dissertação (Mestrado)}
%\tipotrabalho{Tese (Doutorado)}
% O preambulo deve conter o tipo do trabalho, o objetivo, 
% o nome da instituição e a área de concentração 

\preambulo{
Trabalho apresentado ao curso de Gestão do Conhecimento do Instituto Federal de Goiás, como requisito parcial para avaliação da disciplina.}
% ---


% ---
% Configurações de aparência do PDF final

% alterando o aspecto da cor azul
\definecolor{blue}{RGB}{41,5,195}

% informações do PDF
\makeatletter
\hypersetup{
     	%pagebackref=true,
		pdftitle={\@title}, 
		pdfauthor={\@author},
    	pdfsubject={\imprimirpreambulo},
	    pdfcreator={LaTeX with abnTeX2},
		pdfkeywords={abnt}{latex}{abntex}{abntex2}{trabalho acadêmico}, 
		colorlinks=true,       		% false: boxed links; true: colored links
    	linkcolor=blue,          	% color of internal links
    	citecolor=blue,        		% color of links to bibliography
    	filecolor=magenta,      		% color of file links
		urlcolor=blue,
		bookmarksdepth=4
}
\makeatother
% --- 

% ---
% Posiciona figuras e tabelas no topo da página quando adicionadas sozinhas
% em um página em branco. Ver https://github.com/abntex/abntex2/issues/170
\makeatletter
\setlength{\@fptop}{5pt} % Set distance from top of page to first float
\makeatother
% ---

% ---
% Possibilita criação de Quadros e Lista de quadros.
% Ver https://github.com/abntex/abntex2/issues/176
%
\newcommand{\quadroname}{Quadro}
\newcommand{\listofquadrosname}{Lista de quadros}

\newfloat[chapter]{quadro}{loq}{\quadroname}
\newlistof{listofquadros}{loq}{\listofquadrosname}
\newlistentry{quadro}{loq}{0}

% configurações para atender às regras da ABNT
\setfloatadjustment{quadro}{\centering}
\counterwithout{quadro}{chapter}
\renewcommand{\cftquadroname}{\quadroname\space} 
\renewcommand*{\cftquadroaftersnum}{\hfill--\hfill}

\setfloatlocations{quadro}{hbtp} % Ver https://github.com/abntex/abntex2/issues/176
% ---

% --- 
% Espaçamentos entre linhas e parágrafos 
% --- 

% O tamanho do parágrafo é dado por:
\setlength{\parindent}{1.3cm}

% Controle do espaçamento entre um parágrafo e outro:
\setlength{\parskip}{0.2cm}  % tente também \onelineskip

% ---
% compila o indice
% ---
\makeindex
% ---

% ----
% Início do documento
% ----
\begin{document}

% Seleciona o idioma do documento (conforme pacotes do babel)
%\selectlanguage{english}
\selectlanguage{brazil}

% Retira espaço extra obsoleto entre as frases.
\frenchspacing 

% ----------------------------------------------------------
% ELEMENTOS PRÉ-TEXTUAIS
% ----------------------------------------------------------
% \pretextual

% ---
% Capa
% ---
\begin{figure}[h]
	\centering
	\includegraphics[width=1\textwidth]{ufg}
	
\end{figure}
\imprimircapa
% ---

% ---
% Folha de rosto
% (o * indica que haverá a ficha bibliográfica)
% ---
\imprimirfolhaderosto*
% ---

% ---
% Inserir a ficha bibliografica
% ---

% Ficha catalográfica removida conforme solicitação
% ---

% ---
% Inserir errata
% Para incluir uma errata remova o % da linha abaixo
%\include{docs/errata}
% ---

% ---
% Folha de aprovação, dedicatória, agradecimentos e epígrafe removidos conforme solicitação
% ---

% ---
% RESUMOS
% ---

% resumo em português removido conforme solicitação
% \setlength{\absparsep}{18pt} % ajusta o espaçamento dos parágrafos do resumo
% \begin{resumo}
A gestão do conhecimento constitui um pilar estratégico para organizações que buscam manter competitividade e inovação em ambientes dinâmicos. Este trabalho aborda a fundamentação teórica, o histórico de evolução e as aplicações práticas do Confluence, ferramenta de documentação corporativa da Atlassian, como instrumento de gestão do conhecimento. São apresentados os conceitos de base de conhecimento e gestão do conhecimento, distinguindo-se os tipos de conhecimento explícito e tácito. Analisa-se a evolução histórica do Confluence desde sua fundação em 2004 até a incorporação de inteligência artificial em 2024, bem como suas principais funcionalidades e arquitetura conceitual. São discutidas as melhores práticas de implementação, incluindo o princípio de fonte única de verdade, mentalidade aberta, manutenção viva e colaboração dinâmica. Os benefícios da adoção são apresentados em três níveis: operacional, estratégico e cultural. Um estudo de caso em empresa multinacional de blindagem de veículos demonstra resultados expressivos, incluindo redução de 63\% no tempo de onboarding, diminuição de 83\% em incidentes de não conformidade e economia de 90\% no tempo de busca de informações. O trabalho conclui que o Confluence se mostra uma ferramenta estratégica e eficaz, cujo sucesso depende do alinhamento cultural, governança clara, integração ecosistêmica e engajamento contínuo.

\textbf{Palavras-chave}: Gestão do conhecimento. Base de conhecimento. Confluence. Atlassian. Documentação corporativa.
\end{resumo}

% resumo em inglês removido conforme solicitação


% ---
% Listas de ilustrações, quadros e tabelas removidas conforme solicitação
% ---

% ---
% inserir o sumario
% ---
\pdfbookmark[0]{\contentsname}{toc}
\tableofcontents*
\cleardoublepage
% ---



% ----------------------------------------------------------
% ELEMENTOS TEXTUAIS
% ----------------------------------------------------------
\textual

% ----------------------------------------------------------
% Introdução (exemplo de capítulo sem numeração, mas presente no Sumário)
% ----------------------------------------------------------

% ---------------------------------------------------------% ----------------------------------------------------------
% PARTE
% ----------------------------------------------------------
\part*{Introdução}
\chapter{Introdução}

A gestão do conhecimento é um pilar estratégico para organizações que buscam manter competitividade e inovação em ambientes dinâmicos \cite{appvizer2025}. Uma base de conhecimento, quando bem implementada, transforma informações dispersas em ativos organizacionais acessíveis e valorizados. O Confluence, ferramenta de documentação corporativa da Atlassian, emergiu como solução robusta para centralizar, organizar e compartilhar conhecimento em escala empresarial \cite{atlassian2025management}.

Este documento aborda a fundamentação teórica, o histórico de evolução e as aplicações práticas do Confluence como instrumento de gestão do conhecimento, complementado por um estudo de caso que demonstra resultados reais de implementação.

\section{Fundamentação Teórica}

\subsection{Conceitos Essenciais}

Uma base de conhecimento pode ser entendida como um repositório centralizado que funciona como biblioteca online de autoatendimento, contendo informações sobre produtos, serviços, procedimentos e práticas organizacionais \cite{atlassian2025management}. Ela facilita o acesso rápido a informações críticas e reduz a redundância de dados.

Já a gestão do conhecimento consiste em um processo sistêmico de captura, armazenamento, compartilhamento e aplicação do conhecimento organizacional para gerar valor e sustentar vantagem competitiva \cite{appvizer2025}.

\subsection{Tipos de Conhecimento Gerenciados}

\subsubsection{Conhecimento Explícito (Tangível)}

O conhecimento explícito é aquele materializado em documentos, procedimentos, manuais e normativas. Trata-se de um tipo de conhecimento facilmente codificável e estruturado, ideal para armazenamento em sistemas como o Confluence. Alguns exemplos incluem processos operacionais, políticas institucionais e guias técnicos.

\subsubsection{Conhecimento Tácito (Intangível)}

O conhecimento tácito, por sua vez, está enraizado em experiências, habilidades e expertise individual. Ele é difícil de codificar e transferir formalmente, sendo geralmente transmitido por meio de mentoria, colaboração e interação direta. Exemplos desse tipo de conhecimento incluem insights técnicos, boas práticas e aprendizados de projeto.

O Confluence, embora especializado em conhecimento explícito, facilita a captura de conhecimento tácito através de mecanismos colaborativos como comentários, discussões e documentação compartilhada de lições aprendidas \cite{carvalho2023}.

\section{Histórico e Evolução}

\subsection{Linha do Tempo do Confluence}

A evolução do Confluence pode ser compreendida através dos seguintes marcos históricos \cite{eesel2025}:

\begin{table}[htp]
\centering
\caption{Linha do tempo do Confluence}
\label{tab:linha_tempo}
\begin{tabular}{lll} 
\hline
\textbf{Período} & \textbf{Marco} & \textbf{Desenvolvimento} \\ 
\hline
2004 & Fundação & Primeira versão como ferramenta wiki corporativa \\
2011 & Aquisição Atlassian & Integração com Jira; expansão de funcionalidades \\
2015 & Lançamento Cloud & Disponibilidade em nuvem; escalabilidade global \\
2019 & Integração Avançada & APIs robustas; webhooks; automações customizadas \\
2024 & Confluence AI & IA nativa para busca inteligente, resumos e sugestões \\ 
\hline
\end{tabular}
\end{table}

\subsection{Posicionamento no Mercado}

O Confluence consolidou-se como líder em gestão de conhecimento corporativo, competindo com alternativas como SharePoint (Microsoft) e Notion. Sua vantagem diferencial reside na integração nativa com o ecossistema Atlassian (Jira, Bitbucket, Trello) e na abordagem colaborativa baseada no princípio de que tudo deve ser aberto por padrão.

% ----------------------------------------------------------
% ----------------------------------------------------------
% ---------------------------------------------------------% ----------------------------------------------------------
% PARTE
% ----------------------------------------------------------
\part*{Materiais e Métodos}
\chapter{Materiais e Métodos}

\section{A Ferramenta: Atlassian Confluence}

\subsection{Principais Funcionalidades}

O Confluence oferece um conjunto robusto de funcionalidades que o tornam uma ferramenta completa para gestão do conhecimento \cite{atlassian2025management}:

\begin{itemize}
    \item \textbf{Criação e Edição Colaborativa:} Múltiplos usuários podem editar documentos simultaneamente, com controle de versão completo.
    \item \textbf{Armazenamento Centralizado:} Estrutura de espaços (spaces) e páginas hierárquicas para organização lógica do conteúdo.
    \item \textbf{Integração Nativa:} Conexão direta com Jira, Bitbucket, e mais de mil aplicações via APIs.
    \item \textbf{Permissões Granulares:} Controle no nível do usuário sobre acesso a espaços, páginas e documentos.
    \item \textbf{Busca Avançada:} Indexação completa com filtros por tipo, data, autor e palavras-chave.
    \item \textbf{Versionamento:} Rastreamento completo de mudanças com possibilidade de reverter para versões anteriores.
\end{itemize}

\subsection{Arquitetura Conceitual}

O Confluence organiza-se em estruturas hierárquicas que facilitam a navegação e a gestão do conteúdo. Uma instância típica pode ser estruturada da seguinte forma:

A instância do Confluence contém diversos espaços, como por exemplo um espaço de RH (com páginas de Políticas, Procedimentos de Onboarding e Benefícios), um espaço de Engenharia (com Arquitetura de Sistemas, Documentação de APIs e Padrões de Código) e um espaço de Operações (com Runbooks, Postmortems e Procedimentos de Disaster Recovery).

\section{Melhores Práticas de Implementação}

\subsection{Princípios Fundamentais}

De acordo com as recomendações da Atlassian \cite{atlassian2025practices}, os princípios fundamentais para uma implementação eficaz são:

\subsubsection{Fonte Única de Verdade (SSOT)}

É fundamental centralizar todo conhecimento crítico no Confluence para evitar duplicação de informações, versões conflitantes de documentos e confusão sobre qual é o documento oficial.

\subsubsection{Mentalidade Aberta}

Recomenda-se minimizar permissões restritas, incentivar uma cultura de compartilhamento e acesso, além de criar um ambiente de segurança psicológica para experimentação.

\subsubsection{Manutenção Viva}

Para manter a base de conhecimento atualizada, é necessário designar proprietários por espaço e documento, estabelecer calendários de revisão (trimestral ou semestral) e utilizar o versionamento para rastrear a evolução do conteúdo.

\subsubsection{Colaboração Dinâmica}

Deve-se utilizar páginas para brainstorming, implementar workflows de aprovação (especialmente para documentos críticos) e incentivar feedback via comentários, não apenas por e-mails.

\subsection{Estruturação Recomendada}

Para uma implementação eficaz, recomenda-se seguir as seguintes diretrizes:

\begin{enumerate}
    \item \textbf{Templates Padrão:} Criar modelos para tipos comuns de documentos, como Meeting Notes, Decision Records e Runbooks.
    \item \textbf{Convenção de Nomes:} Estabelecer padrão de nomenclatura para facilitar a busca, como por exemplo [PROJETO] Título Descritivo.
    \item \textbf{Taxonomia de Tags:} Implementar um sistema consistente de tags por domínio, prioridade e status.
    \item \textbf{Hierarquia de Espaços:} Alinhar a estrutura aos departamentos, equipes e projetos da organização.
    \item \textbf{Arquivamento:} Definir uma política clara para arquivar conteúdo obsoleto sem deletá-lo.
\end{enumerate}
% ---------------------------------------------------------% ----------------------------------------------------------
% PARTE
% ----------------------------------------------------------
\part*{Resultados e Discussão}
\chapter{Resultados e Discussão}

\section{Benefícios da Adoção}

\subsection{Nível Operacional}

A adoção do Confluence traz benefícios significativos no nível operacional:

\begin{itemize}
    \item \textbf{Redução de Tempo de Busca:} A transição de horas procurando informações em e-mails para minutos utilizando busca indexada representa um ganho expressivo de produtividade.
    \item \textbf{Padronização de Processos:} A documentação única garante consistência operacional em toda a organização.
    \item \textbf{Facilitação de Onboarding:} Novos colaboradores acessam conhecimento consolidado imediatamente, acelerando sua integração.
    \item \textbf{Redução de Duplicação:} Menos retrabalho resulta em maior eficiência geral.
\end{itemize}

\subsection{Nível Estratégico}

No âmbito estratégico, os benefícios incluem:

\begin{itemize}
    \item \textbf{Estímulo à Inovação:} A reutilização de conhecimento permite que as equipes foquem em novos desafios.
    \item \textbf{Retenção de Talentos:} O reconhecimento do conhecimento individual aumenta o engajamento dos colaboradores.
    \item \textbf{Escalabilidade de Processos:} Processos documentados podem ser replicados e aprimorados com maior facilidade.
    \item \textbf{Business Intelligence:} Dados sobre uso da plataforma informam decisões estratégicas.
\end{itemize}

\subsection{Nível Cultural}

Os impactos culturais da adoção também são relevantes:

\begin{itemize}
    \item \textbf{Promoção da Transparência:} Informações acessíveis reduzem silos organizacionais.
    \item \textbf{Maior Engajamento:} Colaboradores sentem-se valorizados quando seu conhecimento é compartilhado.
    \item \textbf{Aprendizado Contínuo:} A cultura de documentação promove reflexão e desenvolvimento profissional.
    \item \textbf{Diversidade Cognitiva:} A exposição a múltiplas perspectivas enriquece a organização.
\end{itemize}

\section{Estudo de Caso: Indústria de Blindagem de Veículos}

\subsection{Contexto}

Segundo \citeonline{nimble2025case}, uma empresa líder global no setor de blindagem veicular enfrentava obstáculos significativos na gestão de suas operações. Em um mercado altamente especializado e competitivo, a organização necessitava garantir processos eficientes, rastreáveis e escaláveis para manter a qualidade e agilidade na entrega de seus serviços.

\subsection{O Desafio}

Conforme relatado por \citeonline{nimble2025case}, a organização operava com sistemas desintegrados, sem comunicação efetiva entre as diferentes etapas do processo produtivo. Esta fragmentação ocasionava gargalos operacionais, comprometia a rastreabilidade dos insumos e limitava a visibilidade sobre a cadeia de suprimentos. Adicionalmente, o gerenciamento manual dos processos de compras e estoque contribuía para atrasos e imprevisibilidade nas entregas.

\subsection{A Solução}

A implementação foi conduzida pela Nimble Evolution, que estruturou um ecossistema digital integrado baseado nas soluções Atlassian \cite{nimble2025case}. A arquitetura proposta fundamentou-se na combinação de Jira Service Management, Jira e Confluence, proporcionando controle abrangente sobre o fluxo produtivo, desde a aquisição de suprimentos até a entrega final ao cliente.

\subsection{Processo de Implementação}

De acordo com \citeonline{nimble2025case}, a implantação seguiu metodologia ágil, permitindo adaptações conforme as necessidades emergentes. As principais intervenções realizadas incluíram:

\begin{itemize}
    \item \textbf{Digitalização dos Processos de Aquisição:} Configuração de fluxos padronizados no Jira Service Management e Jira para solicitação, aprovação e aquisição de materiais, visando reduzir tempos de espera e aumentar a previsibilidade do estoque;
    \item \textbf{Otimização da Cadeia de Suprimentos:} Desenvolvimento de workflows que asseguram integração e rastreabilidade nas compras, fortalecendo o controle financeiro e oferecendo transparência completa sobre pagamentos e fluxos de materiais;
    \item \textbf{Monitoramento em Tempo Real:} Implementação de dashboards e relatórios customizados que permitiram à equipe acompanhar em tempo real o progresso de cada etapa do processo de blindagem;
    \item \textbf{Colaboração Interdepartamental:} Utilização do Jira para estruturar a comunicação entre as equipes de engenharia, produção e suprimentos, minimizando falhas de comunicação e retrabalho;
    \item \textbf{Centralização do Conhecimento Organizacional:} Emprego do Confluence para consolidar documentação de processos, manuais operacionais e melhores práticas, fomentando a cultura de melhoria contínua.
\end{itemize}

\subsection{Resultados Alcançados}

\citeonline{nimble2025case} reportam que a implementação gerou ganhos substanciais:

\begin{itemize}
    \item Diminuição expressiva no tempo de aquisição de insumos, eliminando gargalos e reduzindo o prazo de entrega dos veículos aos clientes;
    \item Incremento na eficiência produtiva através de workflows otimizados, garantindo maior previsibilidade de entregas;
    \item Aprimoramento da rastreabilidade processual, facilitando a identificação e correção proativa de falhas;
    \item Estabelecimento de monitoramento abrangente do ambiente produtivo no Jira, possibilitando detecção e resolução preventiva de incidentes;
    \item Elevação dos índices de satisfação dos clientes devido a prazos mais confiáveis e qualidade padronizada.
\end{itemize}

\subsection{Impacto Estratégico}

Conforme \citeonline{nimble2025case}, a solução implementada consolidou a posição competitiva da empresa no mercado global de blindagem veicular. A automação e integração dos processos não apenas ofereceram vantagem competitiva, mas também ampliaram a escalabilidade e confiabilidade operacional. Tais avanços estratégicos traduziram-se em maior eficiência, melhor atendimento e fortalecimento da reputação organizacional no setor.

\subsection{Perspectivas Futuras}

A organização está expandindo a utilização das soluções Atlassian para outras áreas, incluindo digitalização do atendimento ao cliente e monitoramento em tempo real da qualidade da blindagem \cite{nimble2025case}. O suporte contínuo da Nimble Evolution mantém o projeto em evolução, impulsionando inovação e eficiência operacional.
% ---------------------------------------------------------% ----------------------------------------------------------
% PARTE
% ----------------------------------------------------------
\part*{Conclusão}
\chapter{Conclusão}

O Confluence demonstra ser uma ferramenta estratégica e eficaz para a gestão do conhecimento em organizações complexas. Seu sucesso, contudo, depende não apenas da tecnologia em si, mas fundamentalmente de alguns fatores-chave.

Em primeiro lugar, é necessário um alinhamento cultural que valorize o compartilhamento e a transparência. Sem uma cultura organizacional que incentive a colaboração, mesmo as melhores ferramentas terão dificuldade em gerar os resultados esperados.

Em segundo lugar, uma governança clara é essencial, com políticas bem definidas para manutenção, propriedade e qualidade de conteúdo. Documentos sem donos definidos tendem a se tornar obsoletos rapidamente.

Além disso, a integração ecosistêmica é fundamental. A conexão com fluxos de trabalho existentes, como o Jira, e com outros sistemas corporativos potencializa os benefícios da ferramenta.

Por fim, o engajamento contínuo, por meio de investimento em treinamento, mentoria e reconhecimento, garante que a base de conhecimento permaneça viva e relevante.

O estudo de caso da indústria de blindagem comprova que a implementação adequada do Confluence gera retorno significativo em eficiência operacional, conformidade regulatória e engajamento dos colaboradores. As limitações identificadas são gerenciáveis através de estratégias complementares e tendem a ser mitigadas pelas inovações em inteligência artificial que a Atlassian continua agregando à plataforma.

%\lipsum[31-33]

% ----------------------------------------------------------
% ELEMENTOS PÓS-TEXTUAIS
% ----------------------------------------------------------
\postextual
% ----------------------------------------------------------

% ----------------------------------------------------------
% Referências bibliográficas
% ----------------------------------------------------------
\bibliography{abntex2-modelo-references}

% ----------------------------------------------------------
% Glossário
% ----------------------------------------------------------
%
% Consulte o manual da classe abntex2 para orientações sobre o glossário.
%
%\glossary

% Apêndices e Anexos removidos conforme solicitação

%---------------------------------------------------------------------
% INDICE REMISSIVO
%---------------------------------------------------------------------
\phantompart
\printindex
%---------------------------------------------------------------------

\end{document}
