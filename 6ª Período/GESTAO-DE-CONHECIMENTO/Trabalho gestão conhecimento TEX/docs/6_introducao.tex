\chapter{Introdução}

A gestão do conhecimento é um pilar estratégico para organizações que buscam manter competitividade e inovação em ambientes dinâmicos \cite{appvizer2025}. Uma base de conhecimento, quando bem implementada, transforma informações dispersas em ativos organizacionais acessíveis e valorizados. O Confluence, ferramenta de documentação corporativa da Atlassian, emergiu como solução robusta para centralizar, organizar e compartilhar conhecimento em escala empresarial \cite{atlassian2025management}.

Este documento aborda a fundamentação teórica, o histórico de evolução e as aplicações práticas do Confluence como instrumento de gestão do conhecimento, complementado por um estudo de caso que demonstra resultados reais de implementação.

\section{Fundamentação Teórica}

\subsection{Conceitos Essenciais}

Uma base de conhecimento pode ser entendida como um repositório centralizado que funciona como biblioteca online de autoatendimento, contendo informações sobre produtos, serviços, procedimentos e práticas organizacionais \cite{atlassian2025management}. Ela facilita o acesso rápido a informações críticas e reduz a redundância de dados.

Já a gestão do conhecimento consiste em um processo sistêmico de captura, armazenamento, compartilhamento e aplicação do conhecimento organizacional para gerar valor e sustentar vantagem competitiva \cite{appvizer2025}.

\subsection{Tipos de Conhecimento Gerenciados}

\subsubsection{Conhecimento Explícito (Tangível)}

O conhecimento explícito é aquele materializado em documentos, procedimentos, manuais e normativas. Trata-se de um tipo de conhecimento facilmente codificável e estruturado, ideal para armazenamento em sistemas como o Confluence. Alguns exemplos incluem processos operacionais, políticas institucionais e guias técnicos.

\subsubsection{Conhecimento Tácito (Intangível)}

O conhecimento tácito, por sua vez, está enraizado em experiências, habilidades e expertise individual. Ele é difícil de codificar e transferir formalmente, sendo geralmente transmitido por meio de mentoria, colaboração e interação direta. Exemplos desse tipo de conhecimento incluem insights técnicos, boas práticas e aprendizados de projeto.

O Confluence, embora especializado em conhecimento explícito, facilita a captura de conhecimento tácito através de mecanismos colaborativos como comentários, discussões e documentação compartilhada de lições aprendidas \cite{carvalho2023}.

\section{Histórico e Evolução}

\subsection{Linha do Tempo do Confluence}

A evolução do Confluence pode ser compreendida através dos seguintes marcos históricos \cite{eesel2025}:

\begin{table}[htp]
\centering
\caption{Linha do tempo do Confluence}
\label{tab:linha_tempo}
\begin{tabular}{lll} 
\hline
\textbf{Período} & \textbf{Marco} & \textbf{Desenvolvimento} \\ 
\hline
2004 & Fundação & Primeira versão como ferramenta wiki corporativa \\
2011 & Aquisição Atlassian & Integração com Jira; expansão de funcionalidades \\
2015 & Lançamento Cloud & Disponibilidade em nuvem; escalabilidade global \\
2019 & Integração Avançada & APIs robustas; webhooks; automações customizadas \\
2024 & Confluence AI & IA nativa para busca inteligente, resumos e sugestões \\ 
\hline
\end{tabular}
\end{table}

\subsection{Posicionamento no Mercado}

O Confluence consolidou-se como líder em gestão de conhecimento corporativo, competindo com alternativas como SharePoint (Microsoft) e Notion. Sua vantagem diferencial reside na integração nativa com o ecossistema Atlassian (Jira, Bitbucket, Trello) e na abordagem colaborativa baseada no princípio de que tudo deve ser aberto por padrão.
