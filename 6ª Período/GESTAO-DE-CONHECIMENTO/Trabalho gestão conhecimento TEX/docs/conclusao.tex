\chapter{Conclusão}

O Confluence demonstra ser uma ferramenta estratégica e eficaz para a gestão do conhecimento em organizações complexas. Seu sucesso, contudo, depende não apenas da tecnologia em si, mas fundamentalmente de alguns fatores-chave.

Em primeiro lugar, é necessário um alinhamento cultural que valorize o compartilhamento e a transparência. Sem uma cultura organizacional que incentive a colaboração, mesmo as melhores ferramentas terão dificuldade em gerar os resultados esperados.

Em segundo lugar, uma governança clara é essencial, com políticas bem definidas para manutenção, propriedade e qualidade de conteúdo. Documentos sem donos definidos tendem a se tornar obsoletos rapidamente.

Além disso, a integração ecosistêmica é fundamental. A conexão com fluxos de trabalho existentes, como o Jira, e com outros sistemas corporativos potencializa os benefícios da ferramenta.

Por fim, o engajamento contínuo, por meio de investimento em treinamento, mentoria e reconhecimento, garante que a base de conhecimento permaneça viva e relevante.

O estudo de caso da indústria de blindagem comprova que a implementação adequada do Confluence gera retorno significativo em eficiência operacional, conformidade regulatória e engajamento dos colaboradores. As limitações identificadas são gerenciáveis através de estratégias complementares e tendem a ser mitigadas pelas inovações em inteligência artificial que a Atlassian continua agregando à plataforma.