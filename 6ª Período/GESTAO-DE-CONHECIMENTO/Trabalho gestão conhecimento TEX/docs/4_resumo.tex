\begin{resumo}
A gestão do conhecimento constitui um pilar estratégico para organizações que buscam manter competitividade e inovação em ambientes dinâmicos. Este trabalho aborda a fundamentação teórica, o histórico de evolução e as aplicações práticas do Confluence, ferramenta de documentação corporativa da Atlassian, como instrumento de gestão do conhecimento. São apresentados os conceitos de base de conhecimento e gestão do conhecimento, distinguindo-se os tipos de conhecimento explícito e tácito. Analisa-se a evolução histórica do Confluence desde sua fundação em 2004 até a incorporação de inteligência artificial em 2024, bem como suas principais funcionalidades e arquitetura conceitual. São discutidas as melhores práticas de implementação, incluindo o princípio de fonte única de verdade, mentalidade aberta, manutenção viva e colaboração dinâmica. Os benefícios da adoção são apresentados em três níveis: operacional, estratégico e cultural. Um estudo de caso em empresa multinacional de blindagem de veículos demonstra resultados expressivos, incluindo redução de 63\% no tempo de onboarding, diminuição de 83\% em incidentes de não conformidade e economia de 90\% no tempo de busca de informações. O trabalho conclui que o Confluence se mostra uma ferramenta estratégica e eficaz, cujo sucesso depende do alinhamento cultural, governança clara, integração ecosistêmica e engajamento contínuo.

\textbf{Palavras-chave}: Gestão do conhecimento. Base de conhecimento. Confluence. Atlassian. Documentação corporativa.
\end{resumo}